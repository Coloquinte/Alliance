%%%%%%%%%%%%%%%%%%%%
%
% Data-Path tutorial.
% Author : Jean-Paul CHAPUT
% Modified by czo for Alliance release 4.0 (01/2000)
% TODO : no fully working, needs some adjustements
% $Id: fitpath.tex,v 1.1 2000/01/20 15:19:51 czo Exp $
%
%%%%%%%%%%%%%%%%%%%%%%%%%%%%%%%%%%%%%%%%%%%%%%%%%%%%%%%%%%%%%%%%%%%%%%%%%%%%%%%%


\documentstyle[epsf,a4,here]{article}
%\documentstyle[epsf,a4]{article}
%
%
% New Variables.
%
\def\at{\string @}
\newlength{\sizeindentation}
\setlength{\sizeindentation}{1cm}
%
%
% Predefined Variables.
%
\setlength{\marginparwidth}{0cm}
\setlength{\marginparsep}{0cm}
\setlength{\oddsidemargin}{0.5cm}
\setlength{\evensidemargin}{0.5cm}
\setlength{\textwidth}{14.92cm}
\setlength{\parindent}{\sizeindentation}
%
%
% New Macros.
%
\newcommand{\doubleindent}{\indent\indent}
\newcommand{\forceindent}{\hspace*{\sizeindentation}}
%
%
% Predefined Words and Sentences.
%
\newcommand                {\C} {{\bf C}}
\newcommand           {\column} {{\bf column}}
\newcommand         {\operator} {{\bf operator}}
\newcommand        {\operators} {{\bf operators}}
\newcommand              {\dpr} {{\bf DPR}}
\newcommand              {\lvx} {{\bf LVX}}
\newcommand             {\lynx} {{\bf Lynx}}
\newcommand             {\desb} {{\bf desb}}
\newcommand             {\druc} {{\bf DRuC}}
\newcommand            {\logic} {{\bf Logic}}
\newcommand            {\proof} {{\bf proof}}
\newcommand            {\fpgen} {{\bf FpGen}}
\newcommand           {\genlib} {{\bf GenLib}}
\newcommand           {\asimut} {{\bf ASimuT}}
\newcommand         {\alliance} {{\bf Alliance}}
\newcommand            {\dplib} {{\bf DPLib}}
\newcommand            {\sclib} {{\bf SCLib}}
\newcommand             {\EAST} {{\bf East}}
\newcommand             {\WEST} {{\bf West}}
\newcommand            {\NORTH} {{\bf North}}
\newcommand          {\addaccu} {{\tt addaccu}}
\newcommand              {\fal} {{\tt .al}}
\newcommand             {\fvst} {{\tt .vst}}
\newcommand          {\dprfile} {{\tt .dpr}}
\newcommand         {\desbfile} {{\tt .inf}}
\newcommand        {\sampledpt} {{\tt sample\_dpt}}
\newcommand  {\sampledptsource} {{\tt sample\_dpt.c}}
\newcommand         {\fpgenlib} {{\tt FPGEN\_LIB}}
\newcommand       {\mbkcatalib} {{\tt MBK\_CATA\_LIB}}
\newcommand         {\dpaddiif} {{\tt DP\_ADD2F}}
\newcommand         {\DPADDIIF} {{\tt DP\_ADD2F}}
\newcommand        {\DPMUXIICS} {{\tt DP\_MUX2CS}}
\newcommand            {\DPNUL} {{\tt DP\_NUL}}
\newcommand           {\DPPDFF} {{\tt DP\_PDFF}}
\newcommand         {\DPIMPORT} {{\tt DP\_IMPORT}}
\newcommand       {\DPDEFLOFIG} {{\tt DP\_DEFLOFIG}}
\newcommand       {\DPSAVLOFIG} {{\tt DP\_SAVLOFIG}}
\newcommand          {\netlist} {{\it netlist\,}}
\newcommand           {\layout} {{\it layout\,}}
\newcommand   {\layoutsymbolic} {{\it symbolic layout\,}}
\newcommand         {\symbolic} {{\it symbolic\,}}
\newcommand       {\behavioral} {{\it behavioral\,}}
\newcommand     {\standartcell} {{\it Standart Cell\,}}
\newcommand              {\drc} {{\it DRC\,}}
\newcommand         {\datapath} {{\it data-path\,}}
\newcommand        {\datapaths} {{\it data-paths\,}}
\newcommand        {\gluelogic} {{\it glue logic\,}}
\newcommand     {\datapathcomp} {{\it data-path compiler\,}}
\newcommand            {\slice} {{\it slice\,}}
\newcommand           {\slices} {{\it slices\,}}
\newcommand            {\track} {{\it track\,}}
\newcommand           {\tracks} {{\it tracks\,}}
\newcommand            {\Width} {{\it Width\,}}
\newcommand            {\Slice} {{\it Slice\,}}
\newcommand      {\bourneshell} {{\it Bourne Shell\,}}
%
%
% **************************** Corps du Document *****************************
%
 \begin{document}
 \sloppy %\flushbottom
  %
  %
  % ******************************** Title ********************************
  %
   \begin{titlepage}
   {
     \begin{center}
       {\LARGE {\bf Alliance} {\sc CAD} System {\bf V.R}}\\*[0.3cm]
       {\LARGE e-mail: {\tt cao-vlsi{\at}masi.ibp.fr}}\\*[8cm]
       {\Huge Tutorial for the}\\*[0.4cm]
       {\Huge Data-Path Compiler}\\*[8cm]
       {\Large Jean-Paul C\sc{haput}}\\*[0.2cm]
       {\Large February 24, 1995}
     \end{center}
   }
   \end{titlepage}
  %
  %
  % ************************* Table des Matieres **************************
  %
   \tableofcontents
   \vfill
   \listoffigures
   \newpage
  %
  %
  % **************************** Introduction *****************************
  %
   \section{Introduction}
   \forceindent
     This tutorial aims at teaching how to use \alliance\ {\sc CAD} tools
   dedicated to the design of \datapaths. These tools are the followings~:
   \begin{description}
     \item[\fpgen]
       : \netlist\ capture using textual mode, thanks to a
       set of predefined \C\ functions.
     \item[\dpr]
       : Placer/Router associated with the \datapaths.
   \end{description}
   \forceindent
     From this point on, the environment variable {\bf ALLIANCE\_TOP} represents the root
   directory where the \alliance\ package is located.
  %
  %
  % ************************** Premiere Partie ****************************
  %
   \section{{\it Data-Path}\ basics}
  %
  %
  %  Data-Path Structure
  % =====================
  %
   \subsection{Structure of a \datapath}
   \forceindent
     \datapathcomp\ are tools which are dedicated to the design of
   microproccessor operative units. Inside a \datapath\ the basic
   concept is no more the scalar cell (on 1 bit), but the \operator\
   carrying out a vectorial treatment (on {\bf n bits}).\\
   \indent
     From a logical point of view, an \operator\ is made of a repetition
   of {\bf n} cells more or less identical, each of those executing a one
   bit treatment. It may also contain a cell dedicated to the amplification
   of control signals. Physically, each cell is stacked vertically and
   fill one \slice. The last two \slices are reserved to an amplification
   cell, if needed. It can be talked about either \operator\ or \column,
   this last term receiving a physical acceptance.\\
   \indent
     A \datapath\ is made of \operators\ stacked horizontally. It can be
   represented as a bidimentionnal table in which columns are called
   \operators\ and rows \slices.\\
   \indent
     Figure~\ref{schemadatapath} shows \datapaths\ structure.
   \begin{figure}[hbtp]
     \begin{center}
       \leavevmode\epsfxsize.99\textwidth\epsffile{./sample_dpt-4.eps}
     \end{center}
     \caption{
       \label{schemadatapath}
       {\it \datapaths\ structure.}}
   \end{figure}
  %
  %
  %  Controls and Data Signals 
  % ===========================
  %
   \subsection{Controls and datas signals}
   \label{refctrldataterm}
   \forceindent
     In a \datapath, signals associated with \operator\ input/output buses
   are called {\it datas} signals. They propagate horizontally through the
   \datapath.\\
   \indent
     On the contrary, {\it control} signals are signals which command the
   \operators, for instance, the bus select of a multiplexer ({\it ctrl\_sel}
   of operator \DPMUXIICS). They propagate vertically inside a column.\\
   \indent
     Figure~\ref{ctrldata} shows the difference between {\it control terminals}
   and {\it data terminals}.
   \begin{figure}[H]
     \begin{center}
       \leavevmode\epsfxsize.6\textwidth\epsffile{./sample_dpt-6.eps}
     \end{center}
     \caption{
       \label{ctrldata}
       {\it Data terminals and control terminals.}}
   \end{figure}
   \indent
     Data signals are routed horizontally, over \slices, thanks to 10
   routing \tracks\ ({\tt cf} figure~\ref{schemadatapath}).
   \\*[\bigskipamount]
   {\bf Remark~:}\\*[\bigskipamount]
   \indent
     Only data terminals can be set on any side of the \datapath, using the
   \dprfile\ file. Control terminals always appear on the \NORTH\ side of
   the \datapath, at a predefined place.\\
   \forceindent
     Data terminals sets on the \NORTH\ side of the \datapath\ are also called
   {\it Pseudo control terminals}.
  %
  %
  %  structural Parameters Width and Slices
  % ========================================
  %
   \subsection{Structural Parameters \Width\ and \Slice}
   \begin{minipage}[t]{\textwidth}
     \forceindent
       In some cases we only wish to perform a treatment on a part of data
     bus. We have to use a smaller scaled \operator\ in which empty \slices\
     are to be found. In order to place cells inside the \column, we need
     two values~:
     \begin{tabbing}
       \hspace{\sizeindentation} \= %
       \hspace{1.5cm}      \= %
       \hspace{0.4cm}      \= \kill
       \> \Slice \> : \> The \slice\ were the first cell is put in the
                         \column.\\
       \> \Width \> : \> Effective width of the \operator.
     \end{tabbing}
    %\bigskip
     Default values~:
     \begin{tabbing}
       \hspace{\sizeindentation} \= %
       \hspace{2.5cm}      \= %
       \hspace{0.4cm}      \= \kill
       \> {\tt DEFAULT\_SLICE} \> : \> Null.\\
       \> {\tt DEFAULT\_WIDTH} \> : \> Full width of the \datapath.\\
       \> \> \> (set by the former \DPDEFLOFIG\ function call).
     \end{tabbing}
   \end{minipage}
   \begin{figure}[H]
     \begin{center}
       \leavevmode\epsfysize.9\textwidth\epsffile{./sample_dpt-5.eps}
     \end{center}
     \nopagebreak
     \caption{
       \label{paramstruct}
       {\it Structural parameters \Width\ and \Slice.}}
   \end{figure}
  %
  %
  % ************************** Deuxieme Partie ****************************
  %
   \section{First example~: \sampledpt}
  %
  %
  %  Example Presentation
  % ======================
  %
   \subsection{Presentation of the adder accumulator}
   \forceindent
     The given example in the \datapath\ tutorial is driven from the adder
   accumulator showed in the \addaccu\ tutorial.\\*[\bigskipamount]
   \forceindent
     Differences from \addaccu\ tutorial~:
   \begin{enumerate}
     \samepage
     \item
       Data buses width are set from 4 to 8 bits.
     \item
       In order to show some special features of the \datapathcomp, a zero
       detect has been implemented.
     \item
       The operator implementing sample register elements (\DPPDFF)
       obligatory provide a {\it Write ENable} terminal and dual outputs
       {\it q} and {\it nq}. So we add a {\it ctrl\_wen} terminal to the
       circuit interface and a unused {\it data\_u} signal which doesn't
       appear on the interface (for {\it nq}).
     \item
       As the former operator, the fast adder generator (\DPADDIIF)
       obligatory provide {\it ncout} and {\it nover} outputs.
   \end{enumerate}
   \nopagebreak
     Figure~\ref{schemasample} shows the whole architecture of the adder
   accumulator.
   \begin{figure}[H]
     \vspace*{0.5cm}
     \begin{center}
       \leavevmode\epsfxsize.9\textwidth\epsffile{./sample_dpt-1.eps}
     \end{center}
     \nopagebreak
     \caption{
       \label{schemasample}
       {\it Architecture of the adder accumulator.}}
   \end{figure}
  %
  %
  %  Simplificated Methodology
  % ===========================
  %
   \subsection{Methodology}
   \forceindent
     As we wish not to present the whole design methodology of a circuit but
   only the part related to \datapath, only a partial validation will be
   given.\\
   \nopagebreak
   \indent
     Figure~\ref{methosample}\ describe the partial methodology.\\
   \begin{figure}[hbtp]
     \begin{center}
       \leavevmode\epsfysize.90\textheight\epsffile{./sample_dpt-3.eps}
     \end{center}
     \nopagebreak
     \caption{
       \label{methosample}
       {\it Partial methodology of validation.}}
   \end{figure}
  %
  %
  %  Environment Set-Up
  % ====================
  %
   \subsection{Environment set-up}
   \forceindent
     Before running any tool of the \alliance\ CAD system, some environment
   variables must be set up\footnote{
     Commands are given in \bourneshell.
     }.
   \begin{enumerate}
     \item Input/output formats~:\\
       \fbox{ \shortstack[l]{
         \hspace*{13cm} \\
         {\tt \$ MBK\_IN\_LO=vst} \\
         {\tt \$ MBK\_OUT\_LO=vst} \\
         {\tt \$ MBK\_IN\_PH=ap} \\
         {\tt \$ MBK\_OUT\_PH=ap} \\
         {\tt \$ export MBK\_IN\_LO MBK\_OUT\_LO MBK\_IN\_PH MBK\_OUT\_PH}
       }}
     \item Working directories~:\\
       \fbox{ \shortstack[l]{
         \hspace*{13cm} \\
         {\tt \$ MBK\_WORK\_LIB=.} \\
         {\tt \$ FPGEN\_LIB=./mclib} \\
         {\tt \$ export MBK\_WORK\_LIB FPGEN\_LIB}
       }}
     \item Cell library paths~:\\
       \fbox{ \shortstack[l]{
         \hspace*{13cm} \\
         {\tt \$ MBK\_CATA\_LIB=\$ALLIANCE\_TOP/cells/fitpath/fplib:\$ALLIANCE\_TOP/cells/rsa:\$FPGEN\_LIB} \\
         {\tt \$ export MBK\_CATA\_LIB}
       }}
   \end{enumerate}
  %
  %
  %  FpGen : Netlist Capture, Using Textual Mode
  % =============================================
  %
   \subsection{FpGen : \netlist\ capture}
       The \datapath\ \netlist\ is described thanks to a \C\ source file
     which must contain the following sections~:
     \begin{enumerate}
       \Roman{enumi}
       \setlength{\itemsep}{0cm}
       \samepage
       \item Header files to include~:
       { \tt \begin{verbatim}
#include  <genlib.h>
#include  <fpgen.h>
       \end{verbatim} \rm }
       \pagebreak[1]
       \item Opening the \datapath\ model (\netlist)~:
       { \tt \begin{verbatim}
main()
{
  DP_DEFLOFIG( "sample_dpt", 8, LSB_INDEX_ZERO );
       \end{verbatim} \rm }
       \DPDEFLOFIG\ parameters:
       \begin{tabbing}
         \hspace{\sizeindentation} \= %
         \hspace{2.5cm}            \= %
         \hspace{0.4cm}            \= \kill
         \> {\tt "sample\_dpt"} \> : \> {\bf model name} of the \datapath.\\
         \> {\tt 8} \> : \>  \datapath\ bus wide.\\
         \> {\tt LSB\_INDEX\_ZERO} \> : \> Always set to this value.\\
       \end{tabbing}
       \pagebreak[1]
       \item Terminal declarations~:
       { \tt \begin{verbatim}
  /* Control terminals declarations. */
  DP_LOCON( "ctrl_sel"  , IN   , "ctrl_sel"   );
  DP_LOCON( "ctrl_ck"   , IN   , "ctrl_ck"    );
  DP_LOCON( "ctrl_wen"  , IN   , "ctrl_wen"   );
  DP_LOCON( "ctrl_ncout",   OUT, "ctrl_ncout" );
  DP_LOCON( "ctrl_nover",   OUT, "ctrl_nover" );
  DP_LOCON( "ctrl_zero" ,   OUT, "ctrl_zero"  );

  /* Data terminals declarations. */
  DP_LOCON( "data_a[7:0]" , IN   , "data_a[7:0]" );
  DP_LOCON( "data_b[7:0]" , IN   , "data_b[7:0]" );
  DP_LOCON( "data_s[7:0]" , INOUT, "data_s[7:0]" );

  /* Power supplies terminals. */
  DP_LOCON( "vdd", IN   , "vdd" );
  DP_LOCON( "vss", IN   , "vss" );
       \end{verbatim} \rm }
       \forceindent
         The first string is associated to a terminal model name and the
       second to the internal signal name to which it is connected. The
       behavior of this function is similar to the {\tt LOCON} of \genlib.
       \pagebreak[1]
       \item Instanciation of the various \datapath\ \operators~:
       { \tt \begin{verbatim}
  /* Multiplexer. */
  DP_MUX2CS( "multiplexer", 8, 0,
             "ctrl_sel",
             "data_b[7:0]",
             "data_q[7:0]",
             "data_m[7:0]",
             EOL );

  /* Fast Adder. */
  DP_ADD2F( "adder",
            "data_a[7:0]",
            "data_m[7:0]",
            "ctrl_ncout",
            "ctrl_nover",
            "data_s[7:0]",
            EOL );

  /* Zero Detect. */
  DP_NUL( "zero", 8, 0,
          "data_s[7:0]",
          "ctrl_zero",
          EOL );

  /* Register. */
  DP_PDFF( "memory", 8, 0,
           "ctrl_wen",
           "ctrl_ck",
           "data_s[7:0]",
           "data_q[7:0]",
           "data_u[7:0]",  /* This bus is unused. */
           EOL );
       \end{verbatim} \rm }
       \pagebreak[1]
       \item Complete the model and save to the disk~:
       { \tt \begin{verbatim}
  DP_SAVLOFIG();

  /* A good C program must always terminate by an "exit(0)". */
  exit( 0 );
}
       \end{verbatim} \rm }
     \end{enumerate}
  %
  %
  %  FpGen : Netlist Generation
  % ============================
  %
   \subsection{Generation of the \netlist}
   \forceindent
     After having writen the \C\ source file which describe the \netlist\ 
   and correctly set up the environment, we just have to compile and
   execute the former file. The \fpgen\ script file will do it for us,
   so lets type the command~:
   \\*[\bigskipamount]
   \nopagebreak
   \indent
   \fbox{
     \shortstack[l]{
       \hspace*{13cm}\\
       {\tt fpgen -v sample\_dpt }
   }}\\*[\bigskipamount]
   \nopagebreak
   \indent
     Which normally gives you~:%\\*[\bigskipamount]
   \nopagebreak
   \begin{figure}[H]
     \begin{center}
       \leavevmode\epsfxsize.55\textwidth\epsffile{./trace-fpgen-1.ps}
     \end{center}
   \end{figure}
  %\centerline{\LARGE Trace de FpGen.}\\*[1cm]
   {\bf Back to \mbkcatalib\ and \fpgenlib}
   \\*[\bigskipamount]
   \nopagebreak
   \indent
     Most of the \operators\ avalaible in \fpgen\ are built using
   the leaf cell library {\tt \$ALLIANCE\_TOP/cells/fitpath/fplib}. However, some
   \operators\, as \DPADDIIF, requires additionnal leaf cell
   libraries. These dedicated libraries are noticed in the {\bf UNIX manual
   pages} associated with \operators. In the case of \DPADDIIF\ the needed
   library is {\tt \$ALLIANCE\_TOP/cells/rsa}.\\
   \nopagebreak
   \indent
     So, \mbkcatalib\ must contain~:\\
   \nopagebreak
   \doubleindent
       {\tt \$ALLIANCE\_TOP/cells/fitpath/fplib:\$ALLIANCE\_TOP/cells/rsa}\\*[\bigskipamount]
   \indent
     On the other hand, \fpgen\ creates not only the \datapath\ \netlist\ but
   also the various views (\layout, \behavioral) of the \operators\
   These auxiliary views are stored in the library pointed out
   by the \fpgenlib\ environment variable. In our case, we choose
   {\tt ./mclib}.
  %
  %
  %  DPR : Placing and Routing
  % ===========================
  %
   \subsection{Place and Route}
   \begin{minipage}[t]{\textwidth}
     \forceindent
       To make \datapath\ \layout, run the place and route tool \dpr. The
     command is~:\\*[\bigskipamount]
     \forceindent
     \fbox{
       \shortstack[l]{
         \hspace*{13cm}\\
         {\tt dpr -v -p -r sample\_dpt sample\_dpt}
     }}\\
     \bigskip
   \end{minipage}
   \begin{minipage}[t]{\textwidth}
     \forceindent
       \dpr\ command line arguments\footnotemark~:
     \begin{tabbing}
       \samepage
       \hspace{\sizeindentation} \= %
       \hspace{0.5cm}      \= %
       \hspace{2cm}        \= %
       \hspace{0.4cm}      \= \kill
       \> {\bf -v} \> \> : \> Verbose mode.\\
       \> {\bf -p} \> \> : \> Activate the placement step.\\
       \> {\bf -r} \> {\tt sample\_dpt} \> : \> Ask routing on \netlist\
                                             {\tt sample\_dpt.vst}.\\
       \> \> {\tt sample\_dpt} \> : \> Name of the file that holds the
                                       place and routed \layout.
     \end{tabbing}
   \end{minipage}
   \footnotetext{
     For a more detailed description of \dpr\ command line arguments,
     please refer to the {\bf UNIX} on line manual.
     }
   \forceindent
     This normally produces~:\\*[1cm]
   \nopagebreak
   \begin{figure}[H]
     \begin{center}
       \leavevmode\epsfxsize.55\textwidth\epsffile{./trace-dpr-1.ps}
     \end{center}
   \end{figure}
  %\centerline{\LARGE Trace de DPR.}\\*[1cm]
   {\bf External terminal placement, \dprfile\ file}
   \\*[\bigskipamount]
   \nopagebreak
   \indent
     In addition to the \netlist, \dpr\ attempts to load an optional file
   which contains some instruction about the way to place external data
   terminals
   \footnote{
      For the definitions of data terminals and control terminals, please
      refer~to~\S~\ref{refctrldataterm}.
   }. The \dprfile\ file allows the designer to choose on which side put
   the terminal, and for \EAST\ and \WEST\ sides, which \slice\ and which
   \track.\\
   \nopagebreak
   \indent
     file {\tt sample\_dpt.dpr} used in our example~:\\
   \begin{minipage}[t]{\textwidth}
   \begin{verbatim}
#           Terminal :  Side  :  Slice  :  Track
DP_LOCON   ctrl_zero   NORTH    DEFAULT   DEFAULT
#
#
#           Terminal :  Side  :  Slice  :  Track
DP_LOCON  data_a[7:0]   WEST    DEFAULT   DEFAULT
DP_LOCON  data_b[7:0]   WEST    DEFAULT   DEFAULT
DP_LOCON  data_s[7:0]   EAST    DEFAULT   DEFAULT
#
#
# Number of vertical power refreshment.
DP_POWER  1  50
   \end{verbatim}
   \end{minipage}
   {\bf Back to \mbkcatalib}
   \\*[\bigskipamount]
   \nopagebreak
   \indent
     Place and route tool \dpr\ uses the various views of the operators
   formerly stored by \fpgen\ inside the \fpgenlib\ library. This library
   access path must be present in \mbkcatalib.
   \nopagebreak
   \indent
     At the end, the complete \mbkcatalib\ is:\\
   \doubleindent
   \nopagebreak
       {\tt \$ALLIANCE\_TOP/cells/fitpath/fplib:\$ALLIANCE\_TOP/cells/rsa:\$FPGEN\_LIB}
  %
  %
  %  LYNX, LVX and DESB : Routage Validation
  % =========================================
  %
   \subsection{Validation}
    %
    %
    % Environnement set up for extraction.
    %
     \subsubsection{Environnement set up and validation}
     \nopagebreak
     \forceindent
       All along the generation phasis we have used the \netlist\ format \fvst,
     which stands for {\bf V}{\it HDL}\, {\bf ST}{\it ructural}\,. For the
     validation phasis, we will use the \fal\ format. The \fal\ format is
     dedicated to the extractions steps because it holds, in addition to
     the logical informations (\netlist), some information driven from
     the \layout, such as routing wire capacitances.
     \\*[\bigskipamount]
     \indent
       Setting the new \netlist\ format~:\\
     \nopagebreak
     \indent
     \fbox{ \shortstack[l]{
       \hspace*{13cm} \\
       {\tt \$ MBK\_IN\_LO=al} \\
       {\tt \$ MBK\_OUT\_LO=al}
     }}\\*[\bigskipamount]
     \nopagebreak
     \indent
       Of course, the \layout\ file format remains unchanged.
    %
    %
    % Routage Checking
    %
     \subsubsection{Routage Checking}
     \begin{minipage}[t]{\textwidth}
       \forceindent
         The goal of this step is to ensure that the result of the
       routing phasis is correct. This will be done in three steps~:
       \begin{enumerate}
         \item
           Design rule checking, using the \symbolic\ \drc\ tool \druc.
         \item
           Extration of the gate \netlist\ from the \layout\ using \lynx.
         \item
           Checking coherency between the extracted \netlist\ (\fal\ format)
           and the reference \netlist\ (\fvst\ format).
       \end{enumerate}
       \bigskip
     \end{minipage}
     {\bf Hierarchy management}
     \\*[\bigskipamount]
     \nopagebreak
     \indent
       In some cases, the routing tool \dpr\ is allowed to perform flattening
     in the \layout\ view. As a consequence, \netlist\ extracted from the
     \layout may have a different hierarchy than the reference \netlist.
     In order to process to the validations, we ask to the tools involved
     in the check to work at a gate level.
     \\[\bigskipamount]
     \begin{minipage}[t]{\textwidth}
       {\bf DRuC : \symbolic\ desing rules checking}
       \\*[\bigskipamount]
       \forceindent
       \fbox{ \shortstack[l]{
         \hspace*{13cm} \\
         {\tt \$ druc sample\_dpt}
       }}%\\*[\bigskipamount]
      %\vspace*{1cm}
       \begin{figure}[H]
         \begin{center}
           \leavevmode\epsfxsize.55\textwidth\epsffile{./trace-druc-1.ps}
         \end{center}
       \end{figure}
      %\centerline{\LARGE Trace de DRuC.}%\\*[1cm]
     \end{minipage}
     \begin{minipage}[t]{\textwidth}
       {\bf Lynx : \netlist\ extraction}
       \\*[\bigskipamount]
       \forceindent
       \fbox{ \shortstack[l]{
         \hspace*{13cm} \\
         {\tt \$ lynx -v -f sample\_dpt sample\_dpt\_gates}
       }}%\\*[\bigskipamount]
      %\vspace*{1cm}
       \begin{figure}[H]
         \begin{center}
           \leavevmode\epsfxsize.55\textwidth\epsffile{./trace-lynx-1.ps}
         \end{center}
       \end{figure}
      %\centerline{\LARGE Trace de Lynx.}\\*[1cm]
     \end{minipage}
     \begin{minipage}[t]{\textwidth}
       {\bf LVX : \netlist\ comparisons}
       \\*[\bigskipamount]
       \forceindent
       \fbox{ \shortstack[l]{
         \hspace*{13cm} \\
         {\tt \$ lvx vst al sample\_dpt sample\_dpt\_gates -f}
       }}%\\*[\bigskipamount]
      %\vspace*{1cm}
       \begin{figure}[H]
         \begin{center}
           \leavevmode\epsfxsize.55\textwidth\epsffile{./trace-lvx-1.ps}
         \end{center}
       \end{figure}
      %\centerline{\LARGE Trace de LVX.}
     \end{minipage}
    %
    %
    % Formal Proof.
    %
     \subsubsection{Formal proof}
     \begin{minipage}[t]{\textwidth}
       \forceindent
         The goal of this phasis is to ensure that the \datapath\ we have
       designed here is consistent with its specification. In our case,
       the specification is the {\it VHDL behavioral} view
       {\tt sample\_dpt.vbe}. Three steps are needed~:
       \begin{enumerate}
         \item
            {\bf transistor} \netlist\ extraction from the \layout\ with
            the \lynx\ tool.
         \item
            Restoration of a behavioral view {\tt sample\_dpt\_desb.vbe}
            using the functional abstractor \desb.
         \item
            Formal proof between the behavioral model ({\tt sample\_dpt.vbe})
            and the regenerated model ({\tt sample\_dpt\_desb.vbe}) using
            the formal proofer tool \proof.
       \end{enumerate}
       \bigskip
     \end{minipage}
     \begin{minipage}[t]{\textwidth}
       {\bf Register identification}
       \\*[\bigskipamount]
       \forceindent
         The formal proof tool enforces that the memory elements of both
       behavioral views must have the same names. As these names have
       changed in the \layout, we must rename them. Fortunatly, the
       functional abstractor \desb\ allow us to do so, via an
       auxiliary file \desbfile.\\
       \forceindent
         The file {\tt sample\_dpt.inf} is supplied with the tutorial.
     \end{minipage}
     \begin{minipage}[t]{\textwidth}
       {\bf Lynx : transistor \netlist\ extraction }
       \\*[\bigskipamount]
       \forceindent
       \fbox{ \shortstack[l]{
         \hspace*{13cm} \\
         {\tt \$ lynx -v -t sample\_dpt sample\_dpt}
       }}%\\*[\bigskipamount]
      %\vspace*{1cm}
       \begin{figure}[H]
         \begin{center}
           \leavevmode\epsfxsize.55\textwidth\epsffile{./trace-lynx-2.ps}
         \end{center}
       \end{figure}
      %\centerline{\LARGE Trace de Lynx.}%\\*[1cm]
     \end{minipage}
     \begin{minipage}[t]{\textwidth}
       {\bf DESB : Functional abstraction}\\*[\bigskipamount]
       \forceindent
       \fbox{ \shortstack[l]{
         \hspace*{13cm} \\
         {\tt \$ desb sample\_dpt sample\_dpt\_desb -v -i}
       }}%\\*[\bigskipamount]
      %\vspace*{1cm}
       \begin{figure}[H]
         \begin{center}
           \leavevmode\epsfxsize.55\textwidth\epsffile{./trace-desb-1.ps}
         \end{center}
       \end{figure}
      %\centerline{\LARGE Trace de DESB.}%\\*[1cm]
     \end{minipage}
     \begin{minipage}[t]{\textwidth}
       {\bf Proof : Formal proof}\\*[\bigskipamount]
       \forceindent
       \fbox{ \shortstack[l]{
         \hspace*{13cm} \\
         {\tt \$ proof -d sample\_dpt sample\_dpt\_desb}
       }}%\\*[\bigskipamount]
      %\vspace*{1cm}
       \begin{figure}[H]
         \begin{center}
           \leavevmode\epsfxsize.55\textwidth\epsffile{./trace-proof-1.ps}
         \end{center}
       \end{figure}
      %\centerline{\LARGE Trace de Proof.}
     \end{minipage}
  %
  %
  % ************************* Troisieme Partie ****************************
  %
   \section{Advanced features}
   \begin{minipage}[t]{\textwidth}
     \forceindent
       This section aims to present you some advanced features of \fpgen\ 
     and \dpr\ tools. In the following examples, we do not describe the
     whole procedure given for \sampledpt. We will limits ourselves to show
     differences or novelties from the former methodology.
   \end{minipage}
  %
  %  Placement
  % ===========
  %
   \subsection{Placement}
    %
    %
    % Initial Placement.
    %
     \subsubsection{Initial placement}
     \begin{minipage}[t]{\textwidth}
     \forceindent
         The initial placement of \datapath\ operators is made from left to
       rigth, following the order of instanciation used in the file
       \sampledptsource. We have, starting from the left, the
       followings operators~:
       \begin{enumerate}
         \item
            Multiplexer generated by \DPMUXIICS.
         \item
            Adder generated by \DPADDIIF.
         \item
            Zero detect generated by \DPNUL.
         \item
            D flip-flop generated by \DPPDFF.
       \end{enumerate}
       Figure~\ref{placesample} shows initial placement.
     \end{minipage}
     \vspace*{0.5cm}
     \begin{figure}[H]
       \begin{center}
         \leavevmode\epsfxsize.9\textwidth\epsffile{./sample_dpt-2.eps}
       \end{center}
       \nopagebreak
       \caption{
         \label{placesample}
         {\it Initial placement.}}
     \end{figure}
     \indent
       {\bf The order of instanciation of the operators in \C\ source file
     is meaningful.} A \C\ source file intended to \fpgen/\dpr\ that
     ignores it could produce \underline{non routable configurations}.\\
     \forceindent
       To avoid this kind of problems, we just have to group together the
     instanciations of operators strongly connected.
    %
    %
    % Placement Optimisation.
    %
     \subsubsection{Placement optimization}
     \begin{minipage}[t]{\textwidth}
       \forceindent
         In the case of complex \datapath\ where the designer does not want
       to check precisely the order of operators, placement optimizer inside
       the router tool \dpr\ is avalaible. Roughly, the placement
       optimizer swap operators trying to reduce the total length of
       routing wires. This option is also interresting when the initial
       placement is a non routable configuration. The optimizer is able
       to detect such configuration in which case it attempts to reduce
       the track density to make possible routage.\\
       \forceindent
         File {\tt place\_dpt.*} gives an example of placement optimization.
       \bigskip
     \end{minipage}
     \begin{minipage}[t]{\textwidth}
         Command to invoke \dpr\ in optimization mode~:
       \\*[\bigskipamount]
       \forceindent
       \fbox{
         \shortstack[l]{
           \hspace*{13cm}\\
           {\tt dpr -v -o -r -p place\_dpt place\_dpt}
       }}\\*[\bigskipamount]
       \forceindent
         Arguments given to \dpr\footnotemark~:
       \begin{tabbing}
         \samepage
         \hspace{\sizeindentation} \= %
         \hspace{0.5cm}      \= %
         \hspace{2cm}        \= %
         \hspace{0.4cm}      \= \kill
         \> {\bf -o} \> \> : \> Activate the placement optimizer.\\
       \end{tabbing}
     \end{minipage}
     \footnotetext{
       For a more detailed description of \dpr\ command line arguments,
       please refer to the {\bf UNIX} on line manual.
     }
  %
  %  User Customized Operators
  % ===========================
  %
   \subsection{Designing customized operators}
   \begin{minipage}[t]{\textwidth}
     \forceindent
       In this part we present the features allowing the user to design
     his own \datapath\ operators. To build his operators the user must
     resort to a dedicated leaf cell library~: \dplib\footnotemark.
     \dplib\ offers the same functionnalities than the \standartcell\ library
     \sclib.
   \end{minipage}
   \footnotetext{
     On line manual of \dplib~: {\tt man dplib}, on line manuals of
     cells~: {\tt man n1\_dp, man ms\_dp, ...}
   }\\
   \begin{minipage}[t]{\textwidth}
     \forceindent
       From a methodological point of view, using customized operators is
     unfolding itself into two distinct phases~:
     \begin{enumerate}
       \item
         Making of the operator using one of the following three methods~:
       \begin{enumerate}
         \item
           Explicit building of a column using \genlib. The model name
         must received a {\tt "\_us"} suffix.
         \item
           Synthesis of a block, using \logic. The model name must
         received a {\tt "\_us"} suffix.
         \item
           Buiding a sub-\datapath\ using \fpgen. The model name must not
         have a {\tt "\_us"}, {\tt "\_cl"} or {\tt "\_bk"} suffix.
       \end{enumerate}
       \item
         Instanciation of the newly created operator inside the current
       \datapath\ thanks to the \DPIMPORT\ function.
     \end{enumerate}
   \end{minipage}
    %
    %
    % Environnement Variables.
    %
     \subsubsection{Environnement set up}
     \doubleindent
       We must add to \mbkcatalib\ the access path to the \dplib\ library~:\\
     \nopagebreak
     \indent
     \fbox{ \shortstack[l]{
       \hspace*{13cm} \\
       { \tt \$ MBK\_CATA\_LIB=\$ALLIANCE\_TOP/cells/fitpath/dplib/ecpd10 }\\
       { \tt \$ MBK\_CATA\_LIB=\$MBK\_CATA\_LIB:\$ALLIANCE\_TOP/cells/fitpath/fplib }\\
       { \tt \$ MBK\_CATA\_LIB=\$MBK\_CATA\_LIB:\$ALLIANCE\_TOP/cells/rsa }\\
       { \tt \$ MBK\_CATA\_LIB=\$MBK\_CATA\_LIB:\$FPGEN\_LIB }\\
       { \tt \$ export MBK\_CATA\_LIB }
     } }
    %
    %
    % Single Column Operator.
    %
     \subsubsection{Single Column Operator}
     \begin{minipage}[t]{\textwidth}
       \forceindent
         As an example, let us replace the zero detect provided by \fpgen\ 
       (\DPNUL\ macro-function), by a designer build column.\\
       \forceindent
         The diagram of the zero detect we are going to build is shown in
       figure~\ref{usercolschema}. The files {\tt usercol\_dpt.*} gives
       an example of implementation.
     \end{minipage}
     \begin{figure}[H]
       \begin{center}
         \leavevmode\epsfxsize.4\textwidth\epsffile{./usercol_dpt-2.eps}
       \end{center}
       \nopagebreak
       \caption{
         \label{usercolschema}
         {\it Zero detect diagram.}}
     \end{figure}
     \begin{description}
       \samepage
       \item
         Description of zero detect using \genlib\ language~:
         { \tt \begin{verbatim}
static void  mkZeroDetect()
{
  DEF_LOFIG( "nul_us" );

  LOCON(   "i0[7:0]", IN   ,   "i0[7:0]" );
  LOCON( "zero"     ,   OUT, "zero"      );
  LOCON(  "vdd"     , IN   ,  "vdd"      );
  LOCON(  "vss"     , IN   ,  "vss"      );

  LOINS( "no2_dp", "no2_0", "i0[0]", "i0[1]", "z2_0", "vdd", "vss", 0L );
  LOINS( "no2_dp", "no2_2", "i0[2]", "i0[3]", "z2_1", "vdd", "vss", 0L );
  LOINS( "no2_dp", "no2_4", "i0[4]", "i0[5]", "z2_2", "vdd", "vss", 0L );
  LOINS( "no2_dp", "no2_6", "i0[6]", "i0[7]", "z2_3", "vdd", "vss", 0L );

  LOINS( "na2_dp", "na2_1",  "z2_0",  "z2_1", "z4_0", "vdd", "vss", 0L );
  LOINS( "na2_dp", "na2_5",  "z2_2",  "z2_3", "z4_1", "vdd", "vss", 0L );

  LOINS( "no2_dp", "no2_3",  "z4_0",  "z4_1", "zero", "vdd", "vss", 0L );

  SAVE_LOFIG();
}
         \end{verbatim} \rm }
         \pagebreak[1]
         \forceindent
           As we can see, instance names of the zero detect must conform to
         a specific syntax. The semantic of this syntax is that the suffix
         of the instance name indicates to the \dpr\ placement stage on
         which slice to set each instance. This agreement is recalled on the
         figure~\ref{usercolschema}.
         \pagebreak[3]
       \item
         Modification of the \C\ function describing the \datapath\ \netlist~:
         { \tt \begin{verbatim}
main()
{
  /* Generate the Zero Detect Column. */
  mkZeroDetect();

  /* Open a new Data-Path figure. */
  DP_DEFLOFIG( "usercol_dpt", 8, LSB_INDEX_ZERO );

  /* Interface description. */
  /* ... */

  /* Data-Path netlist description. */ 
  /* Multiplexer ... */
  /* Fast Adder ...  */

  /* Zero Detect.    */
  DP_IMPORT( "nul_us",
             "zero",
             "data_s[7:0]",
             "ctrl_zero",
             EOL );

  /* Register ...    */

  /* Terminate the netlist description, and save on disk. */
  DP_SAVLOFIG();

  exit(0);
}
       \end{verbatim} \rm }
     \end{description}
     \begin{minipage}[t]{\textwidth}
       {\bf Remark~:}\\*[\bigskipamount]
       \forceindent
         The \fpgen\ and \genlib\ languages are made so that only one
       \netlist\ is taken in the course of the description proccess. On the
       other hand, to instanciate the model of zero detect with \DPIMPORT,
       this model must be defined before. As a conclusion, we must call
       the {\tt mkZeroDetect} function before the call to \DPDEFLOFIG.
       \\*[\bigskipamount]
       \forceindent
         The sequence of commands invoked to generate and check this
       example is the same as the one used in \sampledpt.
     \end{minipage}
    %
    %
    % Logical Synthesis of an operator.
    %
     \subsubsection{Logical synthesis of an operator}
     \begin{minipage}[t]{\textwidth}
       \forceindent
         In this section we are going to replace the fast adder provided
       by \DPADDIIF\ by a block generated by logical synthesis. The files
       associated with this example are named {\tt synthese\_dpt.*}.
       \bigskip
     \end{minipage}
     \begin{minipage}[t]{\textwidth}
     {\bf Behavioral description of the adder~:}
       { \tt \begin{verbatim}
ENTITY  adder_us  IS
  PORT(
         a   : in    BIT_VECTOR(7 downto 0);
         b   : in    BIT_VECTOR(7 downto 0);
      cout_n :   out BIT;
      over_n :   out BIT;
         s   :   out BIT_VECTOR(7 downto 0);
       vdd   : in    BIT;
       vss   : in    BIT
      );
END  adder_us;

ARCHITECTURE  behavior_data_flow  OF  adder_us  IS

	SIGNAL cry : BIT_VECTOR(8 downto 0);

BEGIN
  cry(0) <= '0';
  cry(8 downto 1) <=    (a and b)
                     or (a and cry(7 downto 0))
                     or (b and cry(7 downto 0));

  s      <= a xor b xor cry(7 downto 0);
  over_n <= not cry(7);
  cout_n <= not cry(8);

  ASSERT((vdd = '1') and (vss = '0'))
    REPORT "Power supply is missing on adder_us"
    SEVERITY WARNING;
END  behavior_data_flow;
       \end{verbatim} \rm }
       \bigskip
     \end{minipage}
     \noindent
     {\bf Remark~:}\\*[\bigskipamount]
     \nopagebreak
     \indent
       As the terminal ordering is meaningful for instanciation with
     \DPIMPORT, we adopt for the interface of {\tt adder\_us} the same
     order as the one used in the operator provided by \DPADDIIF
     \footnote{
       For curious people, this operator is named {\tt "add2f\_8x8x0l\_bk"}.
     }.\\*[\bigskipamount]
     \begin{minipage}[t]{\textwidth}
       {\bf Environment set up for logical synthesis~:}\\
       \forceindent
       \fbox{ \shortstack[l]{
         \hspace*{13cm} \\
         {\tt \$ MBK\_TARGET\_LIB=\$ALLIANCE\_TOP/cells/fitpath/dplib/ecpd10 }\\
         {\tt \$ MBK\_NAME\_LOG="" }\\
         {\tt \$ export MBK\_TARGET\_LIB MBK\_NAME\_LOG }
       } }%\\*[\bigskipamount]
     \bigskip
     \end{minipage}
     \begin{minipage}[t]{\textwidth}
       {\bf Logic} : Optimization of behavioral equations.
       \\*[\bigskipamount]
       \forceindent
       \fbox{ \shortstack[l]{
         \hspace*{13cm} \\
         {\tt \$ logic -o adder\_us adder\_us\_opt }
       }}%\\*[\bigskipamount]
      %\vspace*{1cm}
       \begin{figure}[H]
         \begin{center}
           \leavevmode\epsfxsize.55\textwidth\epsffile{./trace-logic-1.ps}
         \end{center}
       \end{figure}
      %\centerline{\LARGE Trace de Logic.}%\\*[1cm]
     \end{minipage}
     \begin{minipage}[t]{\textwidth}
       {\bf Logic} : Logical synthesis of the \netlist\ starting from
         previously optimized behavioral description.
       \\*[\bigskipamount]
       \forceindent
       \fbox{ \shortstack[l]{
         \hspace*{13cm} \\
         {\tt \$ logic -s adder\_us\_opt adder\_us }
       }}%\\*[\bigskipamount]
      %\vspace*{1cm}
       \begin{figure}[H]
         \begin{center}
           \leavevmode\epsfxsize.55\textwidth\epsffile{./trace-logic-2.ps}
         \end{center}
       \end{figure}
      %\centerline{\LARGE Trace de Logic.}
     \bigskip
     \end{minipage}
     \pagebreak[3]
     \begin{description}
       \samepage
       \item
         Modification of the \C\ function describing the \datapath\ \netlist~:
         { \tt \begin{verbatim}
main()
{
  /* Open a new Data-Path figure. */
  DP_DEFLOFIG( "synthese_dpt", 8, LSB_INDEX_ZERO );

  /* Interface description. */
  /* ... */

  /* Data-Path netlist description. */ 
  /* Multiplexer ... */

  /* Synthetized Adder. */
  DP_IMPORT( "adder_us",
             "adder",
             "data_a[7:0]",
             "data_m[7:0]",
             "ctrl_ncout",
             "ctrl_nover",
             "data_s[7:0]",
             EOL );

  /* Zero Detect ... */
  /* Register ...    */

  /* Terminate the netlist description, and save on disk. */
  DP_SAVLOFIG();

  exit(0);
}
       \end{verbatim} \rm }
       Further commands are the same as for \sampledpt.
     \end{description}
    %
    %
    % Sub-Data-Path.
    %
     \subsubsection{Hierarchical design~: sub-\datapath}
     \begin{minipage}[t]{\textwidth}
       \forceindent
         As an illustration of the hierarchical design capabilities, we
       have reshape the \netlist\ of the adder accumulator. In this new
       \netlist, the adder (\DPADDIIF) and the zero detect have been put
       together to make a sub-\datapath\ which we call {\tt alu\_dpt}.\\
       \forceindent
         Diagram of figure~\ref{schemahierarchy} shows modifications done to
       the hierarchy. Files related to this example are named
       {\tt "hierarchy\_dpt.*"}.
     \end{minipage}
     \vspace*{0.5cm}
     \begin{figure}[H]
       \begin{center}
         \leavevmode\epsfxsize.90\textwidth\epsffile{./hierarchy_dpt-1.eps}
       \end{center}
       \nopagebreak
       \caption{
         \label{schemahierarchy}
         {\it \netlist\ hierarchy.}}
     \end{figure}
     \begin{description}
       \samepage
       \item
         Description of the sub-\datapath~:
         { \tt \begin{verbatim}
static void  mkSubDP()
{
  /* Open the ALU part of the Data-Path. */
  DP_DEFLOFIG( "alu_dpt", 8, LSB_INDEX_ZERO );

  /* Interface description. */

  /* Control terminals declarations. */
  DP_LOCON( "ctrl_ncout",   OUT, "ctrl_ncout" );
  DP_LOCON( "ctrl_nover",   OUT, "ctrl_nover" );
  DP_LOCON( "ctrl_zero" ,   OUT, "ctrl_zero"  );
  /* Data terminals declarations. */
  DP_LOCON( "data_a[7:0]" , IN   , "data_a[7:0]" );
  DP_LOCON( "data_b[7:0]" , IN   , "data_b[7:0]" );
  DP_LOCON( "data_s[7:0]" , INOUT, "data_s[7:0]" );
  /* Power supplies terminals. */
  DP_LOCON( "vdd", IN   , "vdd" );
  DP_LOCON( "vss", IN   , "vss" );

  /* Data-Path netlist description. */

  /* Fast Adder. */
  DP_ADD2F( "adder",
            "data_a[7:0]",
            "data_b[7:0]",
            "ctrl_ncout",
            "ctrl_nover",
            "data_s[7:0]",
            EOL );

  /* Zero Detect. */
  DP_NUL( "zero", 8, 0,
          "data_s[7:0]",
          "ctrl_zero",
          EOL );


  /* Terminate the netlist description, and save on disk. */
  DP_SAVLOFIG();
}
         \end{verbatim} \rm }
         \pagebreak[1]
       \item
         Modification of the \C\ function describing the \datapath\ \netlist~:
         { \tt \begin{verbatim}
main()
{
  /* Generate the Zero Detect Column. */
  mkSubDP();

  /* Open a new Data-Path figure. */
  DP_DEFLOFIG( "hierarchy_dpt", 8, LSB_INDEX_ZERO );

  /* Interface description. */
  /* ... */

  /* Data-Path netlist description. */ 
  /* Multiplexer ... */

  /* Sub-Data-Path. */
  DP_IMPORT( "alu_dpt",
             "alu",
             "ctrl_ncout",
             "ctrl_nover",
             "ctrl_zero",
             "data_a[7:0]",
             "data_m[7:0]",
             "data_s[7:0]",
             EOL );

  /* Register ...    */

  /* Terminate the netlist description, and save on disk. */
  DP_SAVLOFIG();

  exit(0);
}
         \end{verbatim} \rm }
         \pagebreak[1]
       \item
         Modification of the \dprfile\ file, add the following lines~:
         { \tt \begin{verbatim}
#         Model Name : Iterations : Height : CPC
DP_GLUE    adder_us       5000         8      2
         \end{verbatim} \rm }
         \forceindent
           The placement of \gluelogic\ blocks is automatically performed by
         the router \dpr. The {\tt DP\_GLUE} command of the \dprfile\
         allows a control on the way this placement will be done.\\
         \forceindent
           Parameters meanings~:
         \begin{description}
           \item[Iterations] gives the number of iteration the placement
             algorithm will do.
           \item[Height] height of the block (in \slices).
           \item[CPC] Number of cells per \slice\ (inside a \column).
             The greater the number, the smaller the surface used for the
           block, on the other hand the routing becomes more difficult.
           A good value is generally around 2--3.
         \end{description}
         \pagebreak[3]
         \forceindent
           Placement of \gluelogic\ blocks is systemetically performed at
         each call of the router \dpr. As the placement of blocks coming
         from logical synthesis can take a long time, and in the case
         where we do several successive routing, we can prevent \dpr\ 
         to place the block at each routing phasis. The router will use
         the placement generated at a former iteration. To prevent
         \dpr\ to place a block, just add in the \dprfile~:
         { \tt \begin{verbatim}
#         Model Name
DP_KEEP    adder_us
         \end{verbatim} \rm }
         \forceindent
           Further command are the same as for \sampledpt.
     \end{description}
     \begin{minipage}[t]{\textwidth}
       {\bf Remark~:}\\*[\bigskipamount]
       \forceindent
         By the time the \DPDEFLOFIG\ function saves the \netlist\ on disk,
       this \netlist is automatically flattened at "operator" level. In
       our example, the sub-\datapath\ {\tt "alu\_dpt"} included
       inside {\tt "hierarchy\_dpt"} will be flattened when the whole
       \datapath\ will be terminated.\\
       \forceindent
         This behavior is needed by the fact that the router \dpr\ is not
       able to manage the hierarchy. It only can route a \netlist\ made of
       operators.\\
       \forceindent
         For the sake of clarity regarding the description of the \netlist\ 
       we have chosen to adopt a hierarchical concept at \fpgen\ level.
     \end{minipage}
  %
  %
  % ************************* Quatrieme Partie ****************************
  %
   \newpage
   \appendix
   \section{Files provided with the tutorial}
     Summary of the five examples~:
   \begin{center}
     \begin{tabular}[t]{|c|c|}
       \hline
       fichiers   & exemple                           \\
       \hline
       \hline
       {\tt    sample\_dpt} & first exemple           \\
       {\tt     place\_dpt} & placement optimization  \\
       {\tt   usercol\_dpt} & customized column       \\
       {\tt synthesis\_dpt} & syntesized block        \\
       {\tt hierarchy\_dpt} & hierarchical design     \\
       \hline
     \end{tabular}\\
     \bigskip
   \end{center}
     File associated to each example~:
   \begin{center}
     \begin{tabular}[t]{|c|c|}
       \hline
       extention  & type of file                \\
       \hline
       \hline
       {\tt .sh}  & script in \bourneshell      \\
       {\tt .c}   & \C\ source file for \fpgen  \\
       {\tt .dpr} & auxiliary file for \dpr     \\
       {\tt .inf} & auxiliary file for \desb    \\
       \hline
     \end{tabular}\\
     \bigskip
   \end{center}
   In addition to the five shell scripts, a {\tt Makefile} is also provided.


 \end{document}

